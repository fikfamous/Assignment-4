\documentclass[a4paper,12pt]{article}
\begin{document}


\begin{Huge}
\begin{center}
\begin{normalsize}

\textbf{MAKERERE UNIVERSITY } \\
\textbf{FACULTY OF COMPUTING AND INFORMATICS TECHNOLOGY} \\
\textbf{DEPARTMENT OF COMPUTER SCIENCE} \\
\textbf{BACHELOR OF SCIENCE IN COMPUTER SCIENCE} \\
\textbf{BIT 2207 RESEARCH METHODOLOGY} \\
\textbf{Mr. EARNEST MWEBAZE} \\
\textbf{YEAR 2} \\


\textbf{\sc LUYIMA SHAFIC } \\
\textbf{\sc Reg No: 16/U/6725/PS } \\
\textbf{\sc std No: 216013602}\\
\end{normalsize}
\end{center}
\end{Huge}
\newpage

\newpage

\title{THE IMPACT OF GMAIL IN THE COMMUNICATION SYSTEM}
\author{LUYIMA SHAFIC}      
\renewcommand{\today}{}

\maketitle

\newpage
    
\section{INTRODUCTION}
\paragraph{•}
Gmail is a free, advertising-supported email service developed by Google. Users can access Gmail on the web and using third-party programs that synchronize email content through POP or IMAP protocols. Gmail started as a limited beta release on April 1, 2004, and ended its testing phase on July 7, 2009.
At launch, Gmail had an initial storage capacity offer of one gigabyte per user, a significantly higher amount than competitors offered at the time. Today, the service comes with 15 gigabytes of storage. Users can receive emails up to 50 megabytes in size, including attachments, while they can send emails up to 25 megabytes. 
As of July 2017, Gmail has 1.2 billion active users worldwide,[citation needed] and was the first app on the Google Play Store to hit one billion installations on Android devices. According to a 2014 estimate, 60% of mid-sized US companies, and 92% of startups, were using Gmail.

\section{LITERATURE REVIEW}
\paragraph{•}
I have used both Gmail and Hotmail since 2004 and 1997, respectively. I've transacted over 14,000 emails across both platforms and accumulated over 7 GB of saved data between the 2 services. Until now, I've preferred Gmail for organizing and sending my voluminous messaging. I would go so far as to say Gmail has been the king of webmail services for the last several years for multiple reasons. \cite{Gil}

\paragraph{•}
I recently created a new Gmail account. When I email people now from the new account, instead of coming up as from "Vanessa Snyder" the emails sometimes just say from "GMAIL" (not every time tho). I am afraid people won't read my emails because they look like spam. 
This is a complete nightmare, that I would ignore, but I run many events and this is really causing me some issues. Don't want to leave Gmail :- \cite{Vanessa}



\paragraph{•}
Gmail has several features that I like - in addition to the fact that it is a free service with a large amount of storage space available. I like that Gmail allows you to have multiple accounts - allowing me to have a personal and a business email account - both free of charge - that are easy to switch in between. I am able to navigate between my two Gmail email addresses by clicking one button.  
I dislike that Gmail has one display/set-up for emails and you aren't able to change how emails are displayed in order to customize to your liking. I also dislike the amount of spam that gets bypassed and makes it to my inbox. \cite{Devon}

\begin{thebibliography}{9}
\bibitem{Gil} Paul Gil. \textit{Why Gmail Is Good and Bad}, 
Internet:www.lifewire.com/review-why-gmail-is-good-and-bad-2483272, July 09, 2017  [March 8, 2018].
\bibitem{Vanessa} Vanessa Snyder . \textit{
Why do my emails say they were sent by "Gmail" rather than my name?} 
Internet:www.productforums.google.com/forum/!msg/gmail/ozlU3G8pvmA/gkRHwSH0ZpUJ, October 09, 2014 [March 8, 2018].
\bibitem{Devon} Devon T. \textit{What I like and dislike about Gmail},
 Internet:www.theguardian.com/technology/askjack/2017/jul/27/what-i-like-and-dislike-gmail,February 8, 2018 [March 8, 2018].
\end{thebibliography}
 


\end{document}